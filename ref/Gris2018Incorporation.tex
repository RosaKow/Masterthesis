
\lit{Incorporation of a deformation prior in image reconstruction}{Barbara Gris}{2018}
{\cite{Gris2018Incorporation}. deformation modules, image reconstruction, deformation prior}
{The goal is the reconstruction of images of poor data by using a template image $I_0$ and a deformation prior. Possible applications could be the case when an image at a certain time point is given and the data (obtained by the forward operator $T$) for the following timepoints is poor, while it is known that the desired images transform from the template in a known way.
	The framework is similar to the LDDMM framework but allows only a type of 'intuitive' deformations (depending on the image and application).	
	The large deformation is modelled as the flow of vectorfields specified by deformation modules. The deformtation $\varphi^v$ evolves from the vectorfield $v$ via the flow equation. The space of $v$ is a subspace of all possible vectorfields, denoted by $V$.
	The space $V$ depends on a geometrical descriptor $o\in\Ocal$. An element of $\Ocal$ is specified by the control variable $h\in H$ in a Hilbert space.
	The intuition behind this are 'base motions' associated with a 'geometrical state' of the subject, defining a family of vector fields (exactly those in $V_q$).
	From $q$ and $h$ a vectorfield is generated by the field generator $\zeta\colon \Ocal\times H \longrightarrow V_q, (q,h)\mapsto \zeta_q(h)$.
	The cost $c\colon \Ocal\times H\rightarrow \R$ is a metric and is used as a regularitzation term for the deformation in the energy functional.
	The energy functional 
	\begin{equation}
	J (o,h) = \int_{0}^{1} c_{o_t}(h_t) dt + D( T(\varphi^{\zeta_0(h)}_{t=1}\cdot I_0),d)²
	\label{energy}
	\end{equation}
	is to be minimized. The similarity of the images is measured in the data space.
	In the article well-posedness of the cost as a regularization term is shown.
	
	The deformation Module is defined as $\Mcal = (\Ocal, H, \zeta, c)$. Examples for local translations, contracting-dilating fields and constrained translations generator (CTG) deformation modules are given. It is shown how modules can be compound and can be linearly combined.
	
	As in LDDMM, the shooting equations and momentum are used to find the geodesic of the trajectories in $\Ocal$ minimizing the energy \eqref{energy}.
	
	Experiments are shown with simulated data.
	For this framework, the deformation modules have to be known in advance, which is not the case for real data. Also the size of the Gaussian kernel defining the RKHS $V$ has to be known. 
	The algorithm is not yet satisfying when grey-scale values of the template image are not correct, which could be solved by expanding it to a metamorphosis framework.}