\lit{Introduction to Riemannian and Sub-Riemannian Geometry}{Agrachev, Barilari, Boscain}{2015}{\cite{Agrachev2015Introduction}}
{A Riemannian space is a smooth manifold whose tangent apaces are endowed with Euclidean structures; each tangent space is equipped with its own Euclidean structure that depends smoothly on the point where the tangent space is attached.	
	
A sub-Riemannian space is a smooth manifold with a fixed admissible subspace in any tangent space where admissible subspaces are equipped with Euclidean structures.

Admissible paths are those curves whose velocities are admissible. The distance between points is the infimum of the length of admissible paths connecting the points.

In a control theory spirit: 
admissible paths are solutions to the time-varying ODE
\begin{equation}
\dot{q} (t) = \sum_{i=1}^{k} u_i(t) f_i(q(t)),
\end{equation}
with control functions $u_i$, initial points $q(0)$ and generators $f_i$ of the space of admissible fields.

The Hamiltonian flow on $T^\ast M$  associated to the Hamiltonian H is the sub-Riemannian geodesic flow (normal geodesic). 
Abnormal geodesics belong to the closure of the space of normal geodesics.
}