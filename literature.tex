\documentclass[twoside]{article}
    
    \usepackage{amsmath}
    \usepackage{amsfonts}
    
% Entry command : \lit{<title>}{<author>}{<year>}{<summary>}{<tags>}
    \newcommand{\lit}[5]{%
        \textbf{\Large{#1}} \\%
        %\vspace{0.001cm}\\%
        \large{#2 #3} \\%
        %\vspace{0.03cm}\\%
        \textit{#4} \\%
        \vspace{0.05cm}\\%
        #5\\%
        \vspace{0.5cm}
    }


\newcommand{\Id}{\mathrm{Id}}
\newcommand{\R}{\mathbb{R}}
\newcommand{\Rnm}{(\mathbb{R}^n)^m}
\newcommand{\Ocal}{\mathcal{O}}
\newcommand{\Mcal}{\mathcal{M}}

\begin{document}
	
\lit{Incorporation of a deformation prior in image reconstruction}{Barbara Gris}{2018}
{deformation modules, image reconstruction, deformation prior}
{The goal is the reconstruction of images of poor data by using a template image $I_0$ and a deformation prior. Possible applications could be the case when an image at a certain time point is given and the data (obtained by the forward operator $T$) for the following timepoints is poor, while it is known that the desired images transform from the template in a known way.
	The framework is similar to the LDDMM framework but allows only a type of 'intuitive' deformations (depending on the image and application).	
	The large deformation is modelled as the flow of vectorfields specified by deformation modules. The deformtation $\varphi^v$ evolves from the vectorfield $v$ via the flow equation. The space of $v$ is a subspace of all possible vectorfields, denoted by $V$.
	The space $V$ depends on a geometrical descriptor $o\in\Ocal$. An element of $\Ocal$ is specified by the control variable $h\in H$ in a Hilbert space.
	The intuition behind this are 'base motions' associated with a 'geometrical state' of the subject, defining a family of vector fields (exactly those in $V_q$).
	From $q$ and $h$ a vectorfield is generated by the field generator $\zeta\colon \Ocal\times H \longrightarrow V_q, (q,h)\mapsto \zeta_q(h)$.
	The cost $c\colon \Ocal\times H\rightarrow \R$ is a metric and is used as a regularitzation term for the deformation in the energy functional.
	The energy functional 
	\begin{equation}
	J (o,h) = \int_{0}^{1} c_{o_t}(h_t) dt + D( T(\varphi^{\zeta_0(h)}_{t=1}\cdot I_0),d)²
	\label{energy}
	\end{equation}
	is to be minimized. The similarity of the images is measured in the data space.
	In the article well-posedness of the cost as a regularization term is shown.
	
	The deformation Module is defined as $\Mcal = (\Ocal, H, \zeta, c)$. Examples for local translations, contracting-dilating fields and constrained translations generator (CTG) deformation modules are given. It is shown how modules can be compound and can be linearly combined.
	
	As in LDDMM, the shooting equations and momentum are used to find the geodesic of the trajectories in $\Ocal$ minimizing the energy \eqref{energy}.
	
	Experiments are shown with simulated data.
	For this framework, the deformation modules have to be known in advance, which is not the case for real data. Also the size of the Gaussian kernel defining the RKHS $V$ has to be known. 
	The algorithm is not yet satisfying when grey-scale values of the template image are not correct, which could be solved by expanding it to a metamorphosis framework.}

\lit{Computing large deformation diffeomorphic metric mappings via geodesic flows of diffeomorphisms}
{Beg, Miller, Trouvé, Younes}{2005}
{LDDMM, Euler-Lagrange equation}
{The model is relying on the large deformation approach of Christensen 1996.
 Smoothness is enforced by the norm $\|\cdot\|_V = \|L\cdot\|_{L^2}$ as a regularizer, where $L$ is a differential operator. It is shown that the length of the shortest path in the space of vector fields defines a metric. The Euler-Lagrange equation for the solution of the matching problem is derived. Details of the algorithm are explained: a semi-lagrangian method for integration of the flow equation; the choice of the differential operator $L = -\alpha\nabla^2 + \gamma\Id$ (Cauchy-Navier), where $L^\dag L f = g$. With periodic boundary conditions $L$ is self adjoint.
 In comparison to Christensen's algorithm it is shown that the vector fields vary much less over time.}

\lit{}{Broit, Bajcsy}{}{small deformations approach}
{elastic matching strategy. Transformation $\varphi$ generated by vector field $u$ by $\varphi = id + u$. Goodness of transformation measured by cost $E_2(I_0, I_1, \varphi) = \|I_0\circ\varphi^{-1} - I_1\|_{L_2}$, regularization by $E_1(u) = \|Lu\|_{L_2}^2$, where $L$ is a differential operator. 
Limitation: No constraints for invertability }

\lit{Deformable templates using large deformation kinematics}{Christensen}{1996}{large deformation approach}
{Model of $\varphi$ as the flow of a time-dependent vectorfield.
locally optimal solution, no geodesic}

\lit{Variational Problems on Flows of Diffeomorphisms for Image Matching}{Dupuis, Grenander, Miller}{1997}{optimal control, bayesian estimation, }
{Under the condition $\int_{0}^{\tau} \|v(\cdot,t)\|_L^2 dt < \infty$ (L differential operator), the solution is a diffeomorphism.}

\end{document}